\documentclass[conference]{IEEEtran}
\IEEEoverridecommandlockouts
% The preceding line is only needed to identify funding in the first footnote. If that is unneeded, please comment it out.
\usepackage{cite}
\usepackage{amsmath,amssymb,amsfonts}
\usepackage{algorithmic}
\usepackage{graphicx}
\usepackage{textcomp}
\usepackage{xcolor}
\def\BibTeX{{\rm B\kern-.05em{\sc i\kern-.025em b}\kern-.08em
    T\kern-.1667em\lower.7ex\hbox{E}\kern-.125emX}}
\begin{document}

\title{COMP5048 Assignment 2 (Group)}

\author{\IEEEauthorblockN{Sasha Jenner}
\IEEEauthorblockA{490390494}
\and
\IEEEauthorblockN{Sebastian Kobler}
\IEEEauthorblockA{}
\and
\IEEEauthorblockN{Minerva Predavec}
\IEEEauthorblockA{}
\and
\IEEEauthorblockN{Xingyu Wei}
\IEEEauthorblockA{}
\and
\IEEEauthorblockN{Huayue Zhang}
\IEEEauthorblockA{}
}

\maketitle

\begin{abstract}
\end{abstract}

%\begin{IEEEkeywords}
%component, formatting, style, styling, insert
%\end{IEEEkeywords}

\section{Mobile Device Types}
%What types of mobile devices have been introduced between 1989 and 2013? Can you visually find common characteristics/features to define types of mobile devices?

\subsection{Visualisation}
%Interactive Visualization (or multiple visualizations) allows you to “see” the pieces of information that lead to the answers to the above three questions 1-A/B/C.

\subsection{Design Evaluation}
%Explain how your interactive visualisation has been evaluated and what sort of changes/modifications were applied to derive your final interactive visualisation.

\subsection{Justifications}
%Why certain axes arrangement(s) was(were) used and how they would help to see the information you would need to see,
%Why certain visual variables were used and how they would help to see the information you would need to see,
%Why certain interaction was used and how that would assist analytic processes.

\section{Historical Trends}
%What sorts of “trends” can you observe regarding some types of mobile devices becoming obsolete and being taken over by different/new types?

\subsection{Visualisation}
%Interactive Visualization (or multiple visualizations) allows you to “see” the pieces of information that lead to the answers to the above three questions 1-A/B/C.

\subsection{Design Evaluation}
%Explain how your interactive visualisation has been evaluated and what sort of changes/modifications were applied to derive your final interactive visualisation.

\subsection{Justifications}
%Why certain axes arrangement(s) was(were) used and how they would help to see the information you would need to see,
%Why certain visual variables were used and how they would help to see the information you would need to see,
%Why certain interaction was used and how that would assist analytic processes.

\section{Future Predictions}
%If you were to predict or anticipate a newly emerging mobile device type, what sort of evidence(s) you can visually show to easily detect such emergence? For instance, if you only had access to the data between 1989 and 2010, how would you anticipate new types which became dominant during 2011 and 2013 without seeing the data corresponding to 2011 – 2013?

\subsection{Visualisation}
%Interactive Visualization (or multiple visualizations) allows you to “see” the pieces of information that lead to the answers to the above three questions 1-A/B/C.

\subsection{Design Evaluation}
%Explain how your interactive visualisation has been evaluated and what sort of changes/modifications were applied to derive your final interactive visualisation.

\subsection{Justifications}
%Why certain axes arrangement(s) was(were) used and how they would help to see the information you would need to see,
%Why certain visual variables were used and how they would help to see the information you would need to see,
%Why certain interaction was used and how that would assist analytic processes.

\section*{References}

%\begin{thebibliography}{00}
%\bibitem{b1} G. Eason, B. Noble, and I. N. Sneddon, ``On certain integrals of Lipschitz-Hankel type involving products of Bessel functions,'' Phil. Trans. Roy. Soc. London, vol. A247, pp. 529--551, April 1955.
%\bibitem{b2} J. Clerk Maxwell, A Treatise on Electricity and Magnetism, 3rd ed., vol. 2. Oxford: Clarendon, 1892, pp.68--73.
%\bibitem{b3} I. S. Jacobs and C. P. Bean, ``Fine particles, thin films and exchange anisotropy,'' in Magnetism, vol. III, G. T. Rado and H. Suhl, Eds. New York: Academic, 1963, pp. 271--350.
%\bibitem{b4} K. Elissa, ``Title of paper if known,'' unpublished.
%\bibitem{b5} R. Nicole, ``Title of paper with only first word capitalized,'' J. Name Stand. Abbrev., in press.
%\bibitem{b6} Y. Yorozu, M. Hirano, K. Oka, and Y. Tagawa, ``Electron spectroscopy studies on magneto-optical media and plastic substrate interface,'' IEEE Transl. J. Magn. Japan, vol. 2, pp. 740--741, August 1987 [Digests 9th Annual Conf. Magnetics Japan, p. 301, 1982].
%\bibitem{b7} M. Young, The Technical Writer's Handbook. Mill Valley, CA: University Science, 1989.
%\end{thebibliography}

\end{document}
